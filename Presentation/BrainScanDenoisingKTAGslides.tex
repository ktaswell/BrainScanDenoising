%----------------------------------------------------------------------------------------
%	PACKAGES AND THEMES
%----------------------------------------------------------------------------------------
\documentclass[aspectratio=169,xcolor=dvipsnames]{beamer}
\usetheme{Simple}

\usepackage{graphicx} % Allows including images
\usepackage{booktabs} % Allows the use of \toprule, \midrule and \bottomrule in tables
\usepackage{hyperref}

% begin BibLaTeX section
\usepackage[backend=biber,texencoding=utf8,bibencoding=utf8,style=ieee,maxnames=3,bibwarn=true]{biblatex}
%\renewcommand*{\revsdnamepunct}{} % to eliminate the extra comma between last name and firstmiddle initials
% remember to include the *.bib extension on filenames in the \addbibresource{BibliographicResourceFile} command
\addbibresource{Denoising.bib}
\AtEveryBibitem{\clearlist{language}} % removes unnecessary use of "eng" as language
\AtBeginBibliography{\small}
% end BibLaTeX section

%----------------------------------------------------------------------------------------
%	TITLE PAGE
%----------------------------------------------------------------------------------------

% The title
\title[Denoising with SVD]{Image Denoising via Low-Rank Approximation\\ and Optimal Hard Thresholding}
\subtitle{MAT 167 - Applied Linear Algebra}

\author[Taswell and Garg] {Koby Taswell and Ayush Garg}
\institute[UCD] % Your institution may be shorthand to save space
{
	% Your institution for the title pag
	University of California, Davis 
	\vskip 3pt
}
\date{\today} % Date, can be changed to a custom date


%----------------------------------------------------------------------------------------
%	PRESENTATION SLIDES
%----------------------------------------------------------------------------------------

\begin{document}
	
	\begin{frame}
		% Print the title page as the first slide
		\titlepage
	\end{frame}
	
	\begin{frame}{Overview}
		% Throughout your presentation, if you choose to use \section{} and \subsection{} commands, these will automatically be printed on this slide as an overview of your presentation
		\tableofcontents
	\end{frame}
	
	%------------------------------------------------
	\section{Theory}
	%------------------------------------------------
	
	\begin{frame}{Singular Value Decomposition and Low Rank Approximation}
		\begin{itemize}
			\item Test reference \cite{Golub1987}
		\end{itemize}
	\end{frame}
	
	%------------------------------------------------
	
	\begin{frame}{Blocks of Highlighted Text}
		In this slide, some important text will be \alert{highlighted} because it's important. Please, don't abuse it.
		
		\begin{block}{Block}
			Sample text
		\end{block}
		
		\begin{alertblock}{Alertblock}
			Sample text in red box
		\end{alertblock}
		
		\begin{examples}
			Sample text in green box. The title of the block is ``Examples".
		\end{examples}
	\end{frame}
	
	%------------------------------------------------
	
	\begin{frame}{Multiple Columns}
		\begin{columns}[c] % The "c" option specifies centered vertical alignment while the "t" option is used for top vertical alignment
			
			\column{.45\textwidth} % Left column and width
			\textbf{Heading}
			\begin{enumerate}
				\item Statement
				\item Explanation
				\item Example
			\end{enumerate}
			
			\column{.5\textwidth} % Right column and width
			Lorem ipsum dolor sit amet, consectetur adipiscing elit. Integer lectus nisl, ultricies in feugiat rutrum, porttitor sit amet augue. Aliquam ut tortor mauris. Sed volutpat ante purus, quis accumsan dolor.
			
		\end{columns}
	\end{frame}
	
	%------------------------------------------------
	\section{Easy Application - Kingfisher}
	%------------------------------------------------
	
	\begin{frame}{Table}
		\begin{table}
			\begin{tabular}{l l l}
				\toprule
				\textbf{Treatments} & \textbf{Response 1} & \textbf{Response 2} \\
				\midrule
				Treatment 1         & 0.0003262           & 0.562               \\
				Treatment 2         & 0.0015681           & 0.910               \\
				Treatment 3         & 0.0009271           & 0.296               \\
				\bottomrule
			\end{tabular}
			\caption{Table caption}
		\end{table}
	\end{frame}
	
	%------------------------------------------------
	
	\begin{frame}{Theorem}
		\begin{theorem}[Mass--energy equivalence]
			$E = mc^2$
		\end{theorem}
	\end{frame}
	
	%------------------------------------------------
	
	\begin{frame}{Figure}
		Uncomment the code on this slide to include your own image from the same directory as the template .TeX file.
		%\begin{figure}
		%\includegraphics[width=0.8\linewidth]{test}
		%\end{figure}
	\end{frame}
	
	%------------------------------------------------
	
	\begin{frame}[fragile] % Need to use the fragile option when verbatim is used in the slide
		\frametitle{Citation}
		An example of the \verb|\cite| command to cite within the presentation:\\~
		
		This statement requires citation.
	\end{frame}
	
	%------------------------------------------------
	\section{Hard Application - Medical Imaging}
	%------------------------------------------------
	
	\begin{frame}{References}
		% Beamer does not support BibTeX so references must be inserted manually as below
		\printbibliography
	\end{frame}
	
	%------------------------------------------------
	
	\begin{frame}
		\Huge{\centerline{The End}}
	\end{frame}
	
	%----------------------------------------------------------------------------------------
	
\end{document}